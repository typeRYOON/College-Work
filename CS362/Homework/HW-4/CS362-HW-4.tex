\documentclass{article}

\usepackage{minted}
\usepackage[bookmarksnumbered, colorlinks, plainpages]{hyperref}
\usepackage{fancyhdr}
\usepackage{extramarks}
\usepackage{amsmath}
\usepackage{amsthm}
\usepackage{amsfonts}
\usepackage{tikz}
\usepackage{tabularx}
\usepackage[plain]{algorithm}
\usepackage{algpseudocode}
\usepackage{graphicx}

\graphicspath{ {./img/} }
\usetikzlibrary{automata, positioning}

%
% Basic Document Settings
%

\topmargin=-0.45in
\evensidemargin=0in
\oddsidemargin=0in
\textwidth=6.5in
\textheight=9.0in
\headsep=0.25in

\linespread{1.1}

\pagestyle{fancy}
\lhead{\hmwkAuthorName}
\chead{\hmwkClass\ (\hmwkClassInstructor\ \hmwkClassTime): \hmwkTitle}
\rhead{\firstxmark}
\lfoot{\lastxmark}
\cfoot{\thepage}

\renewcommand\headrulewidth{0.4pt}
\renewcommand\footrulewidth{0.4pt}

\setlength\parindent{0pt}

%
% Create Problem Sections
%

\newcommand{\enterProblemHeader}[1]{
    %\nobreak\extramarks{}{Problem \arabic{#1} continued on next page\ldots}\nobreak{}
    %\nobreak\extramarks{Problem \arabic{#1} (continued)}{Problem \arabic{#1}
    %continued on next page\ldots}\nobreak{}
}

\newcommand{\exitProblemHeader}[1]{
    %\nobreak\extramarks{Problem \arabic{#1} (continued)}{Problem \arabic{#1}
    %continued on next page\ldots}\nobreak{}
    %\stepcounter{#1}
    %\nobreak\extramarks{Problem \arabic{#1}}{}\nobreak{}
}

\setcounter{secnumdepth}{0}
\newcounter{partCounter}
\newcounter{homeworkProblemCounter}
\setcounter{homeworkProblemCounter}{1}
\nobreak\extramarks{}{}\nobreak{}

%
% Homework Problem Environment
%

\newenvironment{homeworkProblem}[1]{
    %\section{Problem \arabic{homeworkProblemCounter}{#1}}
    \setcounter{partCounter}{1}
    %\enterProblemHeader{homeworkProblemCounter}
}{
    %\exitProblemHeader{homeworkProblemCounter}
}

%
% Homework Details
%   - Title
%   - Due date
%   - Class
%   - Section/Time
%   - Instructor
%   - Author
%

\newcommand{\hmwkTitle}{Homework\ 4}
\newcommand{\hmwkDueDate}{March 8, 2024}
\newcommand{\hmwkClass}{CS 362}
\newcommand{\hmwkClassTime}{11:00am}
\newcommand{\hmwkClassInstructor}{Professor Troy}
\newcommand{\hmwkAuthorName}{\textbf{Ryan Magdaleno}}
\newcommand{\hwline}{\begin{center}\line(1,0){358px}\end{center}}

%
% Title Page
%

\title{
    \vspace{2in}
    \textmd{\textbf{\hmwkClass:\ \hmwkTitle}}\\
    \normalsize\vspace{0.1in}\small{Due\ on\ \hmwkDueDate\ at 11:59pm}\\
    \vspace{0.1in}\large{\textit{\hmwkClassInstructor\ \hmwkClassTime}}
    \vspace{3in}
}

\author{\hmwkAuthorName\\\href{mailto:rmagd2@uic.edu}{rmagd2@uic.edu}}
\date{}

\renewcommand{\part}[1]{\textbf{\large Part \Alph{partCounter}}
\stepcounter{partCounter}\\}
%
% Various Helper Commands
%

% Useful for algorithms
\newcommand{\alg}[1]{\textsc{\bfseries \footnotesize #1}}

% For derivatives
\newcommand{\deriv}[1]{\frac{\mathrm{d}}{\mathrm{d}x} (#1)}

% For partial derivatives
\newcommand{\pderiv}[2]{\frac{\partial}{\partial #1} (#2)}

% Integral ds
\newcommand{\dx}{\mathrm{d}x}
\newcommand{\D}[1]{\mathrm{d}#1}

% Image insertion
\newcommand{\img}[2]{\begin{center}\includegraphics[scale=#1]{#2}\end{center}}

% Alias for the Solution section header
\newcommand{\solution}{\textbf{\large Solution}\\}

% Probability commands: Expectation, Variance, Covariance, Bias
\newcommand{\E}{\mathrm{E}}
\newcommand{\Var}{\mathrm{Var}}
\newcommand{\Cov}{\mathrm{Cov}}
\newcommand{\Bias}{\mathrm{Bias}}

\begin{document}

\maketitle

\pagebreak

%%%%%%%%%%%%%%%%%%%%%%%%%%%%%%%%%%%%%%%%%%%%%%%%%%%%%%%%%%%%%%%%%%%%%%%%%%%%%%%%%%%%%%%%%%

\begin{homeworkProblem}{}
    \subsection{1 Report Information}
    \textbf{Title:} Lab Report for Lab 4: Photoresistor (LDR – Light Dependent Resistor)

    \textbf{Name/NetID:} Ryan Magdaleno, rmagd2

    \textbf{Lab Description:} The lab report outlines the creation of a circuit and 
    program utilizing a photoresistor and a 16x2 display. The display indicates the 
    relative amount of light in the room using five predefined text values: dark, 
    partially dark, medium, fully lit, and brightly lit. These values are shown on the 
    top line of the LCD display. The bottom line displays the number of milliseconds 
    since the Arduino was last reset, continuously updating without using the delay() 
    function. Determine the 5 light range values via testing.
    \hwline
\end{homeworkProblem}

%%%%%%%%%%%%%%%%%%%%%%%%%%%%%%%%%%%%%%%%%%%%%%%%%%%%%%%%%%%%%%%%%%%%%%%%%%%%%%%%%%%%%%%%%%

\begin{homeworkProblem}{}
    \vspace{-20pt}\subsection{2 Hardware Required}
    \begin{enumerate}
        \item \vspace{-5pt}
        1x Arduino Uno R3 Board
        \item \vspace{-5pt}
        1x Breadboard
        \item \vspace{-5pt}
        1x 16x2 LCD
        \item \vspace{-5pt}
        1x Photoresistor
        \item \vspace{-5pt}
        1x 10K$\Omega$ Potentiometer
        \item \vspace{-5pt}
        1x 10K$\Omega$ Resistor
        \item \vspace{-5pt}
        1x 220$\Omega$ Resistor
        \item \vspace{-5pt}
        Many wires
    \end{enumerate}
    \hwline
\end{homeworkProblem}

%%%%%%%%%%%%%%%%%%%%%%%%%%%%%%%%%%%%%%%%%%%%%%%%%%%%%%%%%%%%%%%%%%%%%%%%%%%%%%%%%%%%%%%%%%

\begin{homeworkProblem}{}
    \vspace{-20pt}\subsection{3 Circuit Instructions}
    \vspace{-5pt}Here is a step by step guide on how to wire the circuit on the next page.
    \begin{enumerate}
        \item
        To get started with wiring, get your Arduino board and breadboard situated next
        to each other.
        \item \vspace{-5pt}
        Connect a Arduino GND pin to the ground rail on your breadboard.
        \item \vspace{-5pt}
        Next connect your Arduino 5V power pin to the source rail on your breadboard.
        \item \vspace{-5pt}
        On your breadboard, place your photoresistor on two rails, connect one lead to 
        the source rail. Connect the other lead via the 10K$\Omega$ resistor to the
        ground rail. On the grounded lead rail, connect a wire from there to your
        Arduino's analog in A0 pin. This part sends a resistance value based on the
        light level.
        \item \vspace{-5pt}
        Place your 10K$\Omega$ Potentiometer on 3 rails, wire the outer rails to the 
        source and ground rails. \\The middle rail will be wired in a later step, this
        part controls the LCD's contrast.
        \item \vspace{-5pt}
        Place your 16x2 LCD on your breadboard and wire the connections like so: \\
        \begin{tabularx}{0.9\textwidth} { 
            | >{\centering\arraybackslash}X 
            | >{\centering\arraybackslash}X 
            | >{\centering\arraybackslash}X 
            | >{\centering\arraybackslash}X 
            | >{\centering\arraybackslash}X | }
            \hline 16x2 LCD & Wire & Arduino & Potentiometer & Note \\
            \hline VSS (GND) & Ground & GND & -- & -- \\
            \hline VDD (VCC) & Source & 5V & -- & -- \\
            \hline V0 (V0) & Blue & -- & Middle rail & -- \\
            \hline RS (RS) & Orange & 12 & -- & -- \\
            \hline RW (RW) & Ground & GND & -- & -- \\
            \hline E (E) & Yellow & 11 & -- & -- \\
            \hline D4 (DB4) & Green & 5 & -- & -- \\
            \hline D5 (DB5) & Turquoise & 4 & -- & -- \\
            \hline D6 (DB6) & Purple & 3 & -- & -- \\
            \hline D7 (DB7) & Brown & 2 & -- & -- \\
            \hline A (LED) & Source & 5V & -- & via 220$\Omega$ \\
            \hline K (LED) & Ground & GND & -- & -- \\
            \hline
        \end{tabularx}
    \end{enumerate}
\end{homeworkProblem}
\pagebreak

%%%%%%%%%%%%%%%%%%%%%%%%%%%%%%%%%%%%%%%%%%%%%%%%%%%%%%%%%%%%%%%%%%%%%%%%%%%%%%%%%%%%%%%%%%

\begin{homeworkProblem}{}
    \vspace{-20pt}\subsection{4 Diagram}
    \img{0.3}{diagram.png}


    Viewable Simulation: \href{https://www.tinkercad.com/things/9B8VM8YcSko-lab-4?sharecode=UGIQ\_FVNXX0\_Xf9GDLhld34D26ScDwL40DuRshzMJZo}{Lab-4}
    \hwline
\end{homeworkProblem}

%%%%%%%%%%%%%%%%%%%%%%%%%%%%%%%%%%%%%%%%%%%%%%%%%%%%%%%%%%%%%%%%%%%%%%%%%%%%%%%%%%%%%%%%%%

\begin{homeworkProblem}{}
    \vspace{-20pt}\subsection{5 Schematic}
    \img{0.35}{schematic.png}
    \hwline
\end{homeworkProblem}
\pagebreak

%%%%%%%%%%%%%%%%%%%%%%%%%%%%%%%%%%%%%%%%%%%%%%%%%%%%%%%%%%%%%%%%%%%%%%%%%%%%%%%%%%%%%%%%%%

\begin{homeworkProblem}{}
    \vspace{-20pt}\subsection{6 Sample Code}
    \vspace{-5pt}Please install the LiquidCrystal library in your Arduino IDE to run the 
    following code. The goal of the code is to update the LCD while reading the 
    photoresistor's analog values.
    \hwline
    \begin{minted}{cpp}
/* rmagd2-Lab-04.ino */
/* ---------------------------------------------------------------------------------------
 >> Assignment details and provided code are created and owned by Patrick Troy.
 >> University of Illinois Chicago - CS 362, Spring 2024
 >> --------------------------------------------------------------------------------------
 >> File   : rmagd2-Lab-04.ino :: Lab 4 - Photoresistor (LDR – Light Dependent Resistor)
 >> Author : Ryan Magdaleno
 >> UIN/nID: 668523658 (rmagd2)
 >> System : Windows 10 w/ Mingw-w64
 >> TA     : David Levit
 - -                             - -
 >> References used ::
 >> https://docs.wokwi.com/parts/wokwi-photoresistor-sensor
 - -                             - -
 >> File overview ::
 >> This program makes use of a 16x2 LCD along with a photoresistor. The photoresistor
 >> gets the light level and sends its analog output to the arduino to be displayed
 >> on the 16x2 with predefined light level messages. The 16x2 LCD will also be
 >> displaying the time in ms since the start of the program.
--------------------------------------------------------------------------------------- */
// Preprocessor directives ::
#include <LiquidCrystal.h>

// Global variables ::
LiquidCrystal lcd(12, 11, 5, 4, 3, 2);
byte prevLux, curLux;
String luxText[] = {
  "brightly lit    ",
  "fully lit       ",
  "medium          ",
  "partially dark  ",
  "dark            "
};
int ranges[][2] = {
  {501, 2000},
  {201, 500},
  {51, 200},
  {16, 50},
  {0, 15}
};




void setup()
{
  // Set up devices ::
  lcd.begin(16, 2);

  // Set inital values on lcd ::
  prevLux = getLuxIndex();
  lcd.print(luxText[prevLux]);
  lcd.setCursor(0, 1);
  lcd.print("                ");
  lcd.print(millis());
}


// Retrieve corresponding text idex ::
byte getLuxIndex()
{
  int lux = analogRead(A0);
  for (byte i = 0; i < 5; ++i) {
    if (lux >= ranges[i][0] && lux <= ranges[i][1]) {
        return i;
    }
  }
  return -1;
}


void loop()
{
  curLux = getLuxIndex();
  if (prevLux != curLux) {
    lcd.setCursor(0, 0);
    lcd.print(luxText[curLux]);
    prevLux = curLux;
  }
  lcd.print(millis());
  lcd.setCursor(0, 1);
}
    \end{minted}
    \hwline
\end{homeworkProblem}
\pagebreak

%%%%%%%%%%%%%%%%%%%%%%%%%%%%%%%%%%%%%%%%%%%%%%%%%%%%%%%%%%%%%%%%%%%%%%%%%%%%%%%%%%%%%%%%%%

\begin{homeworkProblem}{}
    \subsection{7 References}
    \begin{enumerate}
        \item \vspace{-5pt}
        \href{https://projecthub.arduino.cc/tropicalbean/how-to-use-a-photoresistor-1143fd}
        {Photoresistor tutorial}
        \item \vspace{-5pt}
        \href{https://docs.arduino.cc/learn/electronics/lcd-displays/}
        {16x2 LCD tutorial}
        \item \vspace{-5pt}
        \href{https://www.arduino.cc/reference/en/libraries/liquidcrystal/}
        {LiquidCrystal library documentation}
        \item \vspace{-5pt}
        \href{https://roboticsbackend.com/arduino-potentiometer-complete-tutorial/}
        {Potentiometer tutorial}
    \end{enumerate}
    \hwline
\end{homeworkProblem}

%%%%%%%%%%%%%%%%%%%%%%%%%%%%%%%%%%%%%%%%%%%%%%%%%%%%%%%%%%%%%%%%%%%%%%%%%%%%%%%%%%%%%%%%%%

\begin{homeworkProblem}{}
    \vspace{-20pt}\subsection{8 Experimenting}
    \begin{enumerate}
        \item \vspace{-5pt}
        I determined the ranges by covering the photoresistor with my hands and read the
        lux value via the serial monitor, this was my "dark" values, $\approx$
        0 -- 15 lux.
        \item \vspace{-5pt}
        I then used a flashlight and checked documentation that the max lux value is
        1023, so I used 501 -- 2000 for my "brightly lit" values.
        \item \vspace{-5pt}
        Everything in between was done by moving my flashlight away and letting the
        photoresistor enter darkness, that's how I got my "partially dark", "medium",
        and "fully lit" ranges.
    \end{enumerate}
    \hwline
\end{homeworkProblem}

%%%%%%%%%%%%%%%%%%%%%%%%%%%%%%%%%%%%%%%%%%%%%%%%%%%%%%%%%%%%%%%%%%%%%%%%%%%%%%%%%%%%%%%%%

\begin{homeworkProblem}{}
    \vspace{-20pt}\subsection{9 Conclusion}
    \vspace{-5pt}This was a fun lab, I learned how to use a photoresistor. My final
    project for this class might rely on resistors in case a part of my group's final
    project cannot be achieved. My final project uses solenoids to press keys on a
    keyboard, currently my groupmate is having trouble with the solenoid circuit. The 
    solenoid circuit caught on fire recently and we're looking for alternatives if
    the circuit can't be completed in time for our presentation. I thought of using
    photoresistors to detect the notes falling and sending the data via serial to a
    computer side program for virtual key presses. Writing this report in a how to
    manner also helped me fully understand the materials used in this lab.
    \hwline
\end{homeworkProblem}

%%%%%%%%%%%%%%%%%%%%%%%%%%%%%%%%%%%%%%%%%%%%%%%%%%%%%%%%%%%%%%%%%%%%%%%%%%%%%%%%%%%%%%%%%

\end{document}
