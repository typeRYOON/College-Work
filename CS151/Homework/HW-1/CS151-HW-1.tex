\documentclass[11pt]{article}
\usepackage{minted}
\usepackage{amsfonts, amssymb, amsmath, float}
\usepackage{enumerate, esint, nicefrac, algorithm2e}
\usepackage{enumitem}
\parindent 0px
\title{CS151 :: Homework 1}
\date{February 2, 2022}
\author{Ryan Magdaleno}
\begin{document}
\maketitle

%%%%%%%%%%%%%%%%%%%%%%%%%%%%%%%%%%%%%%%%%%%%%%%%%%%%%%%%%%%%%%%%%%%%%%%%%%%%%%%%%%%%%%%%%

\textbf{Problem 1.} Write each of the following conditional statements and the 
corresponding converse, inverse, and contrapositive as English sentences in the form
"if $p$, then $q$."

\vspace{5px}\textbf{Solution ::} 
\begin{enumerate}[label=1.\arabic* ::]
\item 
When it stays warm for a week, the cherry trees bloom.

Original: \textbf{if} it stays warm for a week, \textbf{then} the cherry trees bloom.\\
Converse: \textbf{if} the cherry trees bloom, \textbf{then} the cherry trees will not
bloom.\\
Inverse: \textbf{if} it does not stay warm for a week, \textbf{then} the cherry trees
will not bloom. \\
Contrapositive: \textbf{if} the cherry trees do not bloom, \textbf{then} it does not stay
warm for a week. \\
\line(1,0){331px}

\item 
Being at least 18 years of age is necessary to get a driver's license.

Original: \textbf{if} you are getting a driver's license, \textbf{then} you are atleast
18 years old. \\
Converse: \textbf{if} you are atleast 18 years old \textbf{then} you are getting a 
driver's license.\\
Inverse: \textbf{if} you are not getting a driver's license, \textbf{then} you are not 
atleast 18 years old.\\
Contrapositive: \textbf{if} you are not atleast 18 years old, \textbf{then} you are not
getting a driver's license.\pagebreak

\item
To be eligible for the honors program, it is sufficient to maintain a 3.4 GPA.

Original: \textbf{if} you maintain a 3.4 GPA, \textbf{then} you are eligible for
the honors program.\\
Converse: \textbf{if} you are eligible for the honors program, \textbf{then} you
maintain a 3.4 GPA.\\
Inverse: \textbf{if} you did not maintain a 3.4 GPA, \textbf{then} you are not eligible
for the honors program.\\
Contrapositive: \textbf{if} you are not eligible for the honors program, \textbf{then}
you not maintain a 3.4 GPA.

\end{enumerate}
\pagebreak

%%%%%%%%%%%%%%%%%%%%%%%%%%%%%%%%%%%%%%%%%%%%%%%%%%%%%%%%%%%%%%%%%%%%%%%%%%%%%%%%%%%%%%%%%

\textbf{Problem 2.} Determine the truth value of each of the following statements if 
the domain consists of all integers. If it is true, find a different domain for which 
it is false. If it is false, find a different domain for which it is true. Justify 
your answer.

\vspace{5px}\textbf{Solution ::}
\begin{enumerate}[label=2.\arabic* ::]
\item 
$\exists x(x^3=-1)$

True. if $x=-1$, then $x^3=-1$. \\
False when $x\ne -1$, then $x^3\ne-1$.

\item
$\exists x(x^2=3)$

False, no whole integer squared equals 3. \\
True when the domain consists of all real numbers, then if $x=\sqrt{3}$
then $x^2=3$.

\item
$\forall x(x+1\le 2x)$

False, if $x=0$, then $0+1\le 2(0)$ is a counterexample. \\
True when $x$ must be $\ge 1, \forall x((x+1\le 2x) \cup (x\ge 1))$.

\item
$\forall x(x^2\ge 0)$

True, negative integers squared are always greater than 0, positive integers squared
must always be great than 0, at $x=0$ it equals 0, so the statement is true. \\
False when the domain includes imaginary numbers, if $x=i$, then $i^2=-1,-1\ge 0$ is
false so this makes the statement false.
\end{enumerate}
\pagebreak

%%%%%%%%%%%%%%%%%%%%%%%%%%%%%%%%%%%%%%%%%%%%%%%%%%%%%%%%%%%%%%%%%%%%%%%%%%%%%%%%%%%%%%%%%

\textbf{Problem 3.} Express each of the following sentences in terms of

$S(x), P(x), Q(x,y)$, quantifiers, and logical operators.

\vspace{5px}\textbf{Solution ::}
\begin{enumerate}[label=3.\arabic* ::]
\item 
Every professor has been asked a question by some student.

$\exists x(S(x)\cap\forall y(P(y)\rightarrow Q(x,y)))$

\item
There is a professor who has never been asked a question by any student.

$\forall x(S(x)\cap\exists y(P(y)\rightarrow \neg Q(x,y)))$

\item
There is a student who has asked a question to exactly one professor.

$\exists x\exists y((S(x)\cap \exists y(P(y)\rightarrow Q(x, y))\cap 
(\forall z((z\ne y)\rightarrow \neg Q(x,z)))))$

\item 
There are two different students who have asked each other a question.

$\exists x\exists y((S(x)\cap S(y))\cap(x\ne y)\cap(Q(x,y)\rightarrow Q(y,x)))$
\end{enumerate}
\pagebreak

%%%%%%%%%%%%%%%%%%%%%%%%%%%%%%%%%%%%%%%%%%%%%%%%%%%%%%%%%%%%%%%%%%%%%%%%%%%%%%%%%%%%%%%%%

\textbf{Problem 4.} Use the laws of propositional logic to prove that the following
 compound propositions are logically equivalent.

\vspace{5px}\textbf{Solution ::}
\begin{enumerate}[label=4.\arabic* ::]
\item 
$(p\cap\neg q)\cup\neg (p\cup q)$ \textbf{and} $\neg q$
\begin{align}
    &(p\cap\neg q)\cup\neg (p\cup q) \\
    &(p\cap\neg q)\cup\neg p\cap\neg q \\
    &(\neg q\cup p)\cap (\neg q\cup\neg q)\cap\neg p \\
    &(\neg q\cup p)\cap (\neg q)\cap\neg p \\
    &(\neg q\cap p)\cup (\neg q\cap\neg q)\cap\neg p \\
    &(\neg q\cap p)\cup\neg q\cap\neg p \\
    &(\neg q\cap p)\cup\neg p\cap\neg q \\
    &(\neg p\cup\neg q)\cap (\neg p\cup\neg q)\cap\neg q \\
    &(\neg p\cup\neg q)\cap\neg q \\
    &\neg q
\end{align}
\textbf{Usage of Classical Propositional Logic Rules ::}

(9) \hspace{5px}used Absorption. \\
(10) used Absorption.

\line(1,0){331px}
\item
$\neg p\rightarrow\neg (q\cup r)$ \textbf{and} $(q\rightarrow p)\cap(r\rightarrow p)$
\begin{align}
    &\neg p\rightarrow\neg (q\cup r) \\
    &p\cup\neg q\cap\neg r \\
    &(p\cup\neg q)\cap (p\cup\neg r) \\
    &(\neg q\cup p)\cap (\neg r\cup p) \\
    &(q\rightarrow p)\cap (r\rightarrow p)
\end{align}

\textbf{Usage of Classical Propositional Logic Rules ::}

(12) used Double Negation and De Morgan's Theorem \\
(13) used Distribution.
\pagebreak

\item
$\neg (p\cup (\neg q\cap (r\rightarrow p)))$ \textbf{and} $\neg p\cap 
(\neg r\rightarrow q)$
\begin{align}
    &\neg (p\cup (\neg q\cap (r\rightarrow p))) \\
    &\neg (p\cup (\neg q\cap (\neg r\cup p))) \\
    &\neg (p\cup (\neg q \cup\neg r)\cap (\neg q\cup p)) \\
    &\neg((p\cap\neg q)\cup(p\cap\neg r)\cap(\neg q\cup p)) \\
    &\neg(p\cap\neg q)\cap\neg(p\cap\neg r)\cup\neg(\neg q\cup p) \\
    &\neg p\cup q\cap\neg p\cup r\cap q \cap\neg p \\
    &\neg p\cup q\cup r\cap q \\
    &\neg p\cap r\cup q \\
    &\neg p\cap(\neg r\rightarrow q)
\end{align}
\line(1,0){331px}

\item
$p\leftrightarrow q$ \textbf{and} $(p\cap q)\cup(\neg p\cap\neg q)$
\begin{align}
    &p\leftrightarrow q \\
    &(p\rightarrow q)\cap(q\rightarrow p) \\
    &(\neg p\rightarrow\neg q)\cap(\neg q\rightarrow\neg p) \\
    &(p\cup\neg q)\cap(q\cup\neg p) \\
    &(p\cap q)\cup(p\cap\neg p)\cup(\neg q\cap q)\cup(\neg q\cap\neg p) \\
    &(p\cap q)\cup (F)\cup (F)\cup(\neg p\cap\neg q) \\
    &(p\cap q)\cup(\neg p\cap\neg q) \\
\end{align}

\textbf{Usage of Classical Propositional Logic Rules ::}

(27) used Contrapositive Rule\\
(29) Expanded $(ac+ad+bc+bd)$.
\pagebreak


\end{enumerate}

\pagebreak

%%%%%%%%%%%%%%%%%%%%%%%%%%%%%%%%%%%%%%%%%%%%%%%%%%%%%%%%%%%%%%%%%%%%%%%%%%%%%%%%%%%%%%%%%

\textbf{Problem 5.} Use the laws of propositional logic to prove that the following
compound propositions are tautologies.

\vspace{5px}\textbf{Solution ::}
\begin{enumerate}[label=5.\arabic* ::]
\item 
$(p\cap q)\rightarrow(q\cup r)$
\begin{align}
    &(p\cap q)\rightarrow(q\cup r) \\
    &\neg(p\cap q)\cup(q\cup r) \\
    &\neg p\cup q\cup(q\cup r) \\
    &\neg p\cup r\cup(q\cup q) \\
    &\neg p\cup r\cup (T) \\
    &\neg p\cup(r\cup T = T) \\
    &\neg p\cup T \\
    &T
\end{align}
\line(1,0){331px}

\item
$((\neg p\rightarrow q)\cap(p\rightarrow r)\cap (q\rightarrow r))\rightarrow r$
\begin{align}
    &((\neg p\rightarrow q)\cap(p\rightarrow r)\cap (q\rightarrow r))\rightarrow r \\
    &\neg((p\cup q)\cap(\neg p\cup r)\cap(\neg q\cup r))\cup r \\
    &\neg(p\cup q)\cup\neg(\neg p\cup r)\cup\neg(\neg q\cup r)\cup r \\
    &\neg p\cap\neg q\cup p\cap\neg r\cup q\cap\neg r\cup r \\
    &\neg p\cap\neg q\cup p\cap\neg r\cup r\cap q\cup r \\
    &\neg p\cap p\cup\neg q\cap(T)\cap q\cup r \\
    &(F)\cup\neg q\cap(T)\cap q\cup r \\
    &(F)\cup\neg q\cap q\cap(T)\cup r \\
    &(F)\cup(F)\cap r\cup(T) \\
    &(F)\cup(T) \\
    &T
\end{align}
\end{enumerate}
\pagebreak

%%%%%%%%%%%%%%%%%%%%%%%%%%%%%%%%%%%%%%%%%%%%%%%%%%%%%%%%%%%%%%%%%%%%%%%%%%%%%%%%%%%%%%%%%

\textbf{Problem 6.} Determine whether each of the following compound propositions is
satisfiable or unsatisfiable. Justify your answer.

\vspace{5px}\textbf{Solution ::}
\begin{enumerate}[label=6.\arabic* ::]
\item 
$(p\cup\neg q)\cap(\neg\cup q)\cap(\neg p\cup\neg q)$ \\
Satifiable when $p = q = F$.
\begin{align}
    (p\cap\neg q) = (F\cup\neg F) &= T \\
    (\neg p\cup q)=(\neg F\cup F) &= T \\
    (\neg p\cup\neg q)=(\neg F\cup\neg F) &= T\\
    T\cap T\cap T &= T \text{ Satifiable.}
\end{align}\line(1,0){331px}

\item
$(p\cup\neg q)\cap(\neg p\cup q)\cap(q\cup r)\cap(q\cup\neg r)\cap(\neg q\cup\neg r)$ \\
Satifiable when $p = q = T$, $r=F$.
\begin{align}
    (p\cup\neg q) = (T\cup\neg T) &= T \\
    (\neg p\cup q) = (\neg T\cup T) &= T \\
    (q\cup r) = (T\cup F) &= T \\
    (q\cup\neg r) = (T\cup\neg F) &= T \\
    (\neg q\cup\neg r) = (\neg T\cup\neg F) &= T \\
    T\cap T\cap T\cap T\cap T &= T \text{ Satifiable.}
\end{align}

\pagebreak
\item
$(p\rightarrow q)\cap(p\rightarrow\neg q)\cap(\neg p\rightarrow q)\cap
(\neg p\rightarrow\neg q)$ \\
Not satisfiable, will use $p = F$, $q = T$.
\begin{align}
    (p\rightarrow q) = (F\rightarrow T) &= T \\
    (p\rightarrow\neg q) = (F\rightarrow\neg T) &= T \\
    (\neg p\rightarrow q) = (\neg F\rightarrow T) &= T \\
    (\neg p\rightarrow\neg q) = (\neg T\rightarrow\neg F) &= F \text{ Not Satifiable.}
\end{align}
\textbf{Justification ::} \\
There's always one statement that will result in false due to $p$ and $q$ both being used
and negated in the statement, and so one sub \\statement will end up false no matter what
truth value of $p$ and $q$.
\line(1,0){331px}

\item
$(\neg p\leftrightarrow\neg q)\cap(\neg p\leftrightarrow q)$ \\
Not satisfiable, will use $p = q = T$.
\begin{align}
    (\neg p\leftrightarrow\neg q) = (\neg T\leftrightarrow\neg T) &= T \\
    (\neg p\leftrightarrow q) = (\neg T\leftrightarrow T) &= F \text{ Not Satifiable.}
\end{align}
\textbf{Justification ::} \\
The first statement will need both variables to be the same to become true for the
biconditional, in the next however that goes against the previous statement because now
one of the statements $q$ is not \\negated causing for this statement to need one truth
value to be $T$ while the other statement to be $F$ causing the whole statement to be not
satisfiable.
\end{enumerate}
\end{document}