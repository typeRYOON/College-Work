\documentclass{article}

\usepackage{minted}
\usepackage[bookmarksnumbered, colorlinks, plainpages]{hyperref}
\usepackage{fancyhdr}
\usepackage{extramarks}
\usepackage{amsmath}
\usepackage{amsthm}
\usepackage{amsfonts}
\usepackage{tikz}
\usepackage[plain]{algorithm}
\usepackage{algpseudocode}
\usepackage{graphicx}



\graphicspath{ {./img/} }
\usetikzlibrary{automata, positioning}

%
% Basic Document Settings
%

\topmargin=-0.45in
\evensidemargin=0in
\oddsidemargin=0in
\textwidth=6.5in
\textheight=9.0in
\headsep=0.25in

\linespread{1.1}

\pagestyle{fancy}
\lhead{\hmwkAuthorName}
\chead{\hmwkClass\ (\hmwkClassInstructor\ \hmwkClassTime): \hmwkTitle}
\rhead{\firstxmark}
\lfoot{\lastxmark}
\cfoot{\thepage}

\renewcommand\headrulewidth{0.4pt}
\renewcommand\footrulewidth{0.4pt}

\setlength\parindent{0pt}

%
% Create Problem Sections
%

\newcommand{\enterProblemHeader}[1]{
    \nobreak\extramarks{}{Problem \arabic{#1} continued on next page\ldots}\nobreak{}
    \nobreak\extramarks{Problem \arabic{#1} (continued)}{Problem \arabic{#1}
    continued on next page\ldots}\nobreak{}
}

\newcommand{\exitProblemHeader}[1]{
    \nobreak\extramarks{Problem \arabic{#1} (continued)}{Problem \arabic{#1}
    continued on next page\ldots}\nobreak{}
    \stepcounter{#1}
    \nobreak\extramarks{Problem \arabic{#1}}{}\nobreak{}
}

\setcounter{secnumdepth}{0}
\newcounter{partCounter}
\newcounter{homeworkProblemCounter}
\setcounter{homeworkProblemCounter}{1}
\nobreak\extramarks{Problem \arabic{homeworkProblemCounter}}{}\nobreak{}

%
% Homework Problem Environment
%

\newenvironment{homeworkProblem}[1]{
    \section{Question \arabic{homeworkProblemCounter}{#1}}
    \setcounter{partCounter}{1}
    \enterProblemHeader{homeworkProblemCounter}
}{
    \exitProblemHeader{homeworkProblemCounter}
}

%
% Homework Details
%   - Title
%   - Due date
%   - Class
%   - Section/Time
%   - Instructor
%   - Author
%

\newcommand{\hmwkTitle}{Homework\ 1}
\newcommand{\hmwkDueDate}{09/18/24}
\newcommand{\hmwkClass}{CS 415}
\newcommand{\hmwkClassTime}{11-12:15pm}
\newcommand{\hmwkClassInstructor}{Professor Tang}
\newcommand{\hmwkAuthorName}{\textbf{Ryan Magdaleno}}
\newcommand{\hwline}{\begin{center}\line(1,0){358px}\end{center}}

%
% Title Page
%

\title{
    \vspace{2in}
    \textmd{\textbf{\hmwkClass:\ \hmwkTitle}}\\
    \normalsize\vspace{0.1in}\small{Due\ on\ \hmwkDueDate\ at 11:59pm}\\
    \vspace{0.1in}\large{\textit{\hmwkClassInstructor\ \hmwkClassTime}}
    \vspace{3in}
}

\author{\hmwkAuthorName\\\href{mailto:rmagd2@uic.edu}{rmagd2@uic.edu}}
\date{}

\renewcommand{\part}[1]{\textbf{\large Part \Alph{partCounter}}
\stepcounter{partCounter}\\}
%
% Various Helper Commands
%

% Useful for algorithms
\newcommand{\alg}[1]{\textsc{\bfseries \footnotesize #1}}

% For derivatives
\newcommand{\deriv}[1]{\frac{\mathrm{d}}{\mathrm{d}x} (#1)}

% For partial derivatives
\newcommand{\pderiv}[2]{\frac{\partial}{\partial #1} (#2)}

% Integral ds
\newcommand{\dx}{\mathrm{d}x}
\newcommand{\D}[1]{\mathrm{d}#1}

% Image insertion
\newcommand{\img}[2]{\begin{center}\includegraphics[scale=#1]{#2}\end{center}}

% Alias for the Solution section header
\newcommand{\solution}{\textbf{\large Solution}\\}

% Probability commands: Expectation, Variance, Covariance, Bias
\newcommand{\E}{\mathrm{E}}
\newcommand{\Var}{\mathrm{Var}}
\newcommand{\Cov}{\mathrm{Cov}}
\newcommand{\Bias}{\mathrm{Bias}}

\begin{document}

\maketitle

\pagebreak

%%%%%%%%%%%%%%%%%%%%%%%%%%%%%%%%%%%%%%%%%%%%%%%%%%%%%%%%%%%%%%%%%%%%%%%%%%%%%%%%%%%%%%%%%

\begin{homeworkProblem}{}
    \begin{enumerate}
        \item What is the goal of computer vision?\\
        To give computers the ability to process, analyze, and interpret 
        information from images/videos. Computers in general follow set instructions so
        giving a computer something as subjective as "what are you looking at" is simple
        to us humans but much more complex when it comes to computers, the goal is to give
        computers that ability to extract information from an input image/video.
        \item Please list three computer vision tasks (for example, face detection) and 
        their respective applications.
        \begin{enumerate}
            \item \textbf{Pose Estimation:} Computers determine the spatial position/
            orientation of objects/body parts from some input image/video. This is
            used in athletics, VTubing, face filters, motion capture.
            \item \textbf{Scene Reconstruction:} Computer identifies key geometric /
            topological aspects of a image/set of images and reconstruct a 3D space
            that aims to be a replica of said input image(s). This is used in
            medicine, take a scan of someone's veins, we can construct a 3D mapping
            of this now. Used in crime scene reconstruction.
            \item \textbf{Segmentation:} A Computer takes an imput image and segments the
            image into distinct object classifications. Take an image of a dog and cat,
            the computer will identify portions of the image that are of a dog and 
            portions that are of a cat, everything else will be classified as other.
            This is used in self driving car programs, a auto car program needs to know
            what portions of the image are the road, cars, pedestrians, etc.
        \end{enumerate}
        \item What is a (digital) RGB image?\\
        A 3D matrix of elements each with three channels representing red, green, and 
        blue. There may also be a alpha channel.
    \end{enumerate}
    \hwline\vspace{-20px}
\end{homeworkProblem}

%%%%%%%%%%%%%%%%%%%%%%%%%%%%%%%%%%%%%%%%%%%%%%%%%%%%%%%%%%%%%%%%%%%%%%%%%%%%%%%%%%%%%%%%%

\begin{homeworkProblem}{}
    \begin{enumerate}
        \item Please briefly describe the process of linear filtering.
        Because an image is a 3D matrix of intensity values, we can apply operations on
        it to modify said image. We use something known as a kernel which will be the
        matrix at that will be applied to a "region". We can apply this kernel on
        all regions of a image by doing something like convolution or cross-correlation.
        What we end up with is an image that has been filtered on every pixel, using the 
        same kernel.
        \item What are the commonality and difference between (cross) correlation and 
        convolution?\\
        \textbf{Commonalities:}
        \vspace{-10px}\begin{enumerate}
            \item Both are applied to each region in the input image (if padded).
            \item Same amount of iterations for each operation.
            \item Both apply filters/kernels.
        \end{enumerate}
        \textbf{Differences:}
        \vspace{-10px}\begin{enumerate}
            \item Correlation uses the kernel as is, convolution flips it vertically and 
            horizontally.
            \item Output images are different.
        \end{enumerate}
    \end{enumerate}

    \hwline\vspace{-20px}
\end{homeworkProblem}
\pagebreak

%%%%%%%%%%%%%%%%%%%%%%%%%%%%%%%%%%%%%%%%%%%%%%%%%%%%%%%%%%%%%%%%%%%%%%%%%%%%%%%%%%%%%%%%%

\begin{homeworkProblem}{}
    Below are a 3x3 grayscale input image (left) and a 3x3 kernel (right). Please manually
    perform correlation and convolution. Zero padding should be used to make the size of 
    the output image the same as that of the input image.
    \[
        \begin{bmatrix}
            1 & 0 & 2 \\
            2 & 2 & 1 \\
            2 & 1 & 0
        \end{bmatrix}
        \hspace{30px}
        \begin{bmatrix}
            2 & 1 & 1 \\
            1 & 2 & 0 \\
            0 & 0 & 1
        \end{bmatrix}
    \]
    \hwline\solution
    \textbf{Correlation:}
    \[
        \begin{bmatrix}
            0 & 0 & 0 & 0 & 0\\
            0 & 1 & 0 & 2 & 0\\
            0 & 2 & 2 & 1 & 0\\
            0 & 2 & 1 & 0 & 0\\
            0 & 0 & 0 & 0 & 0
        \end{bmatrix}
        *
        \begin{bmatrix}
            2 & 1 & 1\\
            1 & 2 & 0\\
            0 & 0 & 1
        \end{bmatrix}
        =
        \begin{bmatrix}
            4 & 2 & 4\\
            6 & 10 & 6\\
            8 & 11 & 6
        \end{bmatrix}
    \]
    \vspace{-10px}
    \small{\begin{align*}
        0*2 + 0*1 + 0*1 + 0*1 + 1*2 + 0*0 + 0*0 + 2*0 + 2*1 &= 4\\
        0*2 + 0*1 + 0*1 + 1*1 + 0*2 + 2*0 + 2*0 + 2*0 + 1*1 &= 2\\
        0*2 + 0*1 + 0*1 + 0*1 + 2*2 + 0*0 + 2*0 + 1*0 + 0*1 &= 4\\
        0*2 + 1*1 + 0*1 + 0*1 + 2*2 + 2*0 + 0*0 + 2*0 + 1*1 &= 6\\
        1*2 + 0*1 + 2*1 + 2*1 + 2*2 + 1*0 + 2*0 + 1*0 + 0*1 &= 10\\
        0*2 + 2*1 + 0*1 + 2*1 + 1*2 + 0*0 + 1*0 + 0*0 + 0*1 &= 6\\
        0*2 + 2*1 + 2*1 + 0*1 + 2*2 + 1*0 + 0*0 + 0*0 + 0*1 &= 8\\
        2*2 + 2*1 + 1*1 + 2*1 + 1*2 + 0*0 + 0*0 + 0*0 + 0*1 &= 11\\
        2*2 + 1*1 + 0*1 + 1*1 + 0*2 + 0*0 + 0*0 + 0*0 + 0*1 &= 6
    \end{align*}}
    \hwline
    \textbf{Convolution:}
    \[
        \begin{bmatrix}
            0 & 0 & 0 & 0 & 0\\
            0 & 1 & 0 & 2 & 0\\
            0 & 2 & 2 & 1 & 0\\
            0 & 2 & 1 & 0 & 0\\
            0 & 0 & 0 & 0 & 0
        \end{bmatrix}
        *
        \begin{bmatrix}
            1 & 0 & 0\\
            0 & 2 & 1\\
            1 & 1 & 2
        \end{bmatrix}
        =
        \begin{bmatrix}
            8 & 8 & 7\\
            10 & 9 & 3\\
            5 & 4 & 2
        \end{bmatrix}
    \]
    \vspace{-10px}
    \small{\begin{align*}
        0*1 + 0*0 + 0*0 + 0*0 + 1*2 + 0*1 + 0*1 + 2*1 + 2*2 &= 8\\
        0*1 + 0*0 + 0*0 + 1*0 + 0*2 + 2*1 + 2*1 + 2*1 + 1*2 &= 8\\
        0*1 + 0*0 + 0*0 + 0*0 + 2*2 + 0*1 + 2*1 + 1*1 + 0*2 &= 7\\
        0*1 + 1*0 + 0*0 + 0*0 + 2*2 + 2*1 + 0*1 + 2*1 + 1*2 &= 10\\
        1*1 + 0*0 + 2*0 + 2*0 + 2*2 + 1*1 + 2*1 + 1*1 + 0*2 &= 9\\
        0*1 + 2*0 + 0*0 + 2*0 + 1*2 + 0*1 + 1*1 + 0*1 + 0*2 &= 3\\
        0*1 + 2*0 + 2*0 + 0*0 + 2*2 + 1*1 + 0*1 + 0*1 + 0*2 &= 5\\
        2*1 + 2*0 + 1*0 + 2*0 + 1*2 + 0*1 + 0*1 + 0*1 + 0*2 &= 4\\
        2*1 + 1*0 + 0*0 + 1*0 + 0*2 + 0*1 + 0*1 + 0*1 + 0*2 &= 2
    \end{align*}}
    \hwline
\end{homeworkProblem}
\pagebreak

%%%%%%%%%%%%%%%%%%%%%%%%%%%%%%%%%%%%%%%%%%%%%%%%%%%%%%%%%%%%%%%%%%%%%%%%%%%%%%%%%%%%%%%%%

\begin{homeworkProblem}{}
    \textbf{P1:}
    Implement the convolution operator. Directly calling a convolution or filtering 
    function from any library is prohibited. You can use the linear filtering code in our 
    code tutorial as a template (available in Blackboard) or build your own code from 
    scratch. You are encouraged to implement your own Gaussian function. Please use 
    padding to keep the image size unchanged.
    \begin{enumerate}
        \item Use convolution to apply mean, Gaussian (std=1), and sharping filters to 
        lena.png.
        \item Try different kernel sizes: 3x3, 5x5, and 7x7.
    \end{enumerate}
    \textbf{P2:}
    Implement the median filter (same requirement as P1). To keep the image size 
    unchanged, you may simply ignore the pixels outside the input image when calculating 
    the median value of a patch.
    \begin{enumerate}
        \item Apply both mean and median filters to art.png.
        \item Try different kernel sizes: 3x3, 5x5, 7x7, and 9x9
    \end{enumerate}
    \textbf{P3:}
    Self-study the filter2D function in OpenCV. Use it to perform Gaussian filtering on 
    lena.png with different kernel sizes (3x3, 5x5, and 7x7). Are the results the same as 
    those obtained by your implementation in P1?

    \hwline\solution
    For \textbf{P1} and \textbf{P2}, i've included the full code below in this PDF for 
    you to check, you can also find it within the bundled main.py file. I will include 
    an image showcasing this program's output image on the next page.\\

    \textbf{P3:} The images were nearly identical, the only difference was that the edges
    of my program's output images had black lines which is from the zero padding I did.
    OpenCV's filter2D function did not have this, most likely it was implemented using
    something like reflect padding. The reason mine has black lines on the edges is
    because my convolution function, specifically apply\_filter uses those zeroes from
    the padded kernel in its calculations.
    \hwline
    \pagebreak
    \img{1.2}{E:/谷桜/proj/uni/CS415/Homework/HW-1/img/P3.png}
    \hwline
    \pagebreak
    \begin{minted}{python3}
#  main.py
''' --------------------------------------------------------------------------------------
>> Assignment details and provided code are created and owned by Wang Tei.
>> University of Illinois at Chicago - CS 415, Fall 2024
>> --------------------------------------------------------------------------------------
>> File   :: main.py
>> Course :: CS 415 (42844), FA24
>> Author :: Ryan Magdaleno (rmagd2)
>> System :: Windows 10 w/ Python 3.11.3
- -                             - -
>> File overview ::
>> This program implements convolution and is able to apply multiple filters onto
>> images. This program makes heavy use of cv2 and numpy to achieve this goal.
>> There are multiple filters to use, including a custom kernel you can modify.
>> You can mix and match filters, along with using cv2's filter2D function instead
>> of this program's implementation (cv2 is way faster). The mean filter can't be
>> used by cv2's filter2D function.
- -                             - -
>> Usage:
>> ret = convolution("lena.png", 7, ["cv2", "mean", "median"])
>> cv2 must be in the 0th index if you want to use that, you can use other filters
>> in combination after index 0.
>> Meaning: Apply 7x7 mean and median filters onto "lena.png" using cv2.filter2D.
>>
>> ret = convolution("art.png", 3, ["mean", "median"])
>> Meaning: Apply 3x3 mean and median filters onto "art.png" using apply_filter.
-------------------------------------------------------------------------------------- '''
# Module Imports ::
from os import path as opath
from typing import List
from sys import argv
import numpy as np
import cv2 as cv

# Mean kernel is all ones divided by k_size^2 :: - -                                   - -
def mean_kernel(k_size: int) -> np.ndarray:
    return np.ones((k_size, k_size), dtype=np.float32) / (k_size * k_size)












# Use np.fromfunction to apply a 2D gaussian function on each (x,y) pos.
# G(x, y)  = 1/( 2(pi))(sigma)^2 ) * exp( -( ( x-u_x )^2 + ( y-u_y )^2 ) / 2(sigma)^2 )
# (sigma)  = std, in this case always 1.0
# u_x, u_y = center of kernel: (k_size - 1) / 2 
# I(Ryan), personally used this link to help make this implementation:
# http://www.devanddep.com/tutorial/numpy/how-to-generate-2-d-gaussian-array-using-numpy.html
def gaussian_kernel(k_size: int, std: float=1.0) -> np.ndarray:
    ret = np.fromfunction(
        lambda x, y:
            (1 / (2 * np.pi * std ** 2))
            * np.exp(
                -((x - (k_size - 1) / 2) ** 2 + (y - (k_size - 1) / 2) ** 2)
                / (2 * std ** 2)
            ),
        (k_size, k_size)
    )
    return ret / np.sum(ret) # Normalized
































# The sharpen kernel emphasizes differences in adjacent pixel values  :: - -           - -
def sharpening_kernel(k_size: int) -> np.ndarray:
    if k_size == 3:
        ret = np.array([
            [ 0, -1,  0],
            [-1,  5, -1],
            [ 0, -1,  0]
        ], dtype=np.float32)
    elif k_size == 5:
        ret = np.array([
            [ 0,  0, -1,  0,  0],
            [ 0, -1, -1, -1,  0],
            [-1, -1, 13, -1, -1],
            [ 0, -1, -1, -1,  0],
            [ 0,  0, -1,  0,  0]
        ], dtype=np.float32)
    elif k_size == 7:
        ret = np.array([
            [ 0,  0,  0, -1,  0,  0,  0],
            [ 0,  0, -1, -1, -1,  0,  0],
            [ 0, -1, -1, -1, -1, -1,  0],
            [-1, -1, -1, 21, -1, -1, -1],
            [ 0, -1, -1, -1, -1, -1,  0],
            [ 0,  0, -1, -1, -1,  0,  0],
            [ 0,  0,  0, -1,  0,  0,  0]
        ], dtype=np.float32)
    elif k_size == 9:
        ret = np.array([
            [ 0,  0,  0,  0, -1,  0,  0,  0,  0],
            [ 0,  0,  0, -1, -1, -1,  0,  0,  0],
            [ 0,  0, -1, -1, -1, -1, -1,  0,  0],
            [ 0, -1, -1, -1, -1, -1, -1, -1,  0],
            [-1, -1, -1, -1,  35, -1, -1, -1, -1],
            [ 0, -1, -1, -1, -1, -1, -1, -1,  0],
            [ 0,  0, -1, -1, -1, -1, -1,  0,  0],
            [ 0,  0,  0, -1, -1, -1,  0,  0,  0],
            [ 0,  0,  0,  0, -1,  0,  0,  0,  0]
        ], dtype=np.float32)
    return ret / np.sum(ret) # Normalized










# Custom kernel, make sure to match k_size passed in convolution function :: - -       - -
def custom_kernel(k_size: int) -> np.ndarray:
    ret = np.array([
        [0, 0, 0],
        [0, 0, 0],
        [0, 0, 0]
    ])
    if ret.shape[0] != k_size or ret.shape[1] != k_size:
        exit(" > Custom kernel shape is not k_size * k_size")
    return ret

# Padding utility function :: - -                                                      - -
def pad_image(img: np.ndarray, pad_amount: int) -> np.ndarray:
    return np.pad(
        img,
        (
            (pad_amount, pad_amount),
            (pad_amount, pad_amount),
            (0, 0)
        )
    )

# Median implementation using np.sort and np.ndarray.flatten :: - -                    - -
def median(region: np.ndarray) -> float:
    sorted_region = np.sort(region.flatten())
    n = len(sorted_region)
    if n & 0x1:
        return sorted_region[n // 2]
    else:
        return (sorted_region[n // 2 - 1] + sorted_region[n // 2]) / 2

# Similar to apply_filter, but uses median function here instead :: - -                - -
def median_filter(img: np.ndarray, k_size: int) -> np.ndarray:
    img_height, img_width, img_channels = img.shape
    output = np.zeros_like(img, dtype=np.float32)
    img = pad_image(img, k_size // 2)

    for i in range(img_height):
        for j in range(img_width):
            region = img[i:i+k_size, j:j+k_size]
            for c in range(img_channels):
                output[i, j, c] = median(region[:, :, c])

    return output





# Apply given kernel onto image via convolution :: - -                                 - -
def apply_filter(
    img: np.ndarray,
    kernel: np.ndarray,
    k_size: int,
    use_cv: bool) -> np.ndarray:

    if use_cv:
        return cv.filter2D(img, -1, kernel.astype(np.float32))

    img_height, img_width, img_channels = img.shape
    output = np.zeros_like(img, dtype=np.float32)
    img = pad_image(img, k_size // 2)

    for i in range(img_height):
        for j in range(img_width):
            region = img[i:i+k_size, j:j+k_size]
            for c in range(img_channels):
                output[i, j, c] = np.sum(region[:, :, c] * kernel)

    return output




























# Load kernel / image and apply a filter onto the image via convolution :: - -         - -
def convolution(input_name: str, k_size: int, filters: List[str]) -> str:
    img = cv.imread(input_name, cv.IMREAD_COLOR)
    if img is None:
        return f"Image load error (cv2): \"{input_name}\""

    if k_size < 3 or k_size > 9:
        return f"Kernel size must be 3, 5, 7, or 9: {k_size}"

    if not k_size & 0x1:
        return f"Even kernel size: {k_size}"

    use_cv = False
    if filters and filters[0] == "cv2":
        print(" > Using cv2 filter2D.")
        filters = filters[1:]
        use_cv = True

    if filters is None:
        return f"Add some filter strings"

    kernels = {
        "mean": mean_kernel,
        "custom": custom_kernel,
        "gaussian": gaussian_kernel,
        "sharpening": sharpening_kernel
    }
    for filter in filters:
        if filter == "median":
            img = median_filter(img, k_size)
        elif filter in kernels:
            img = apply_filter(img, np.flip(kernels[filter](k_size)), k_size, use_cv)
        else:
            return f"Invalid kernel filter string passed: {filter}"
        print(f" > {k_size}x{k_size} {filter} filter done.")

    if use_cv:
        filters.insert(0, "cv2")
    ext = opath.splitext(input_name)
    cv.imwrite(f"{ext[0]}_{k_size}x{k_size}_{'-'.join(filters)}{ext[1]}", img)
    return None

# Program entrypoint :: - -                                                            - -
if __name__ == "__main__":
    ret_str = convolution("lena.png", k_size=3, filters=["cv2", "mean", "gaussian"])
    if ret_str:
        print(f" > {ret_str}")
    \end{minted}
\end{homeworkProblem}
\end{document}