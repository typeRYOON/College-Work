\documentclass{article}

\usepackage[bookmarksnumbered, colorlinks, plainpages]{hyperref}
\usepackage{fancyhdr}
\usepackage{extramarks}
\usepackage{amsmath}
\usepackage{amsthm}
\usepackage{amsfonts}
\usepackage{tikz}
\usepackage[plain]{algorithm}
\usepackage{algpseudocode}
\usepackage{graphicx}

\graphicspath{ {./img/} }
\usetikzlibrary{automata, positioning}

%
% Basic Document Settings
%

\topmargin=-0.45in
\evensidemargin=0in
\oddsidemargin=0in
\textwidth=6.5in
\textheight=9.0in
\headsep=0.25in

\linespread{1.1}

\pagestyle{fancy}
\lhead{\hmwkAuthorName}
\chead{\hmwkClass\ (\hmwkClassInstructor\ \hmwkClassTime): \hmwkTitle}
\rhead{\firstxmark}
\lfoot{\lastxmark}
\cfoot{\thepage}

\renewcommand\headrulewidth{0.4pt}
\renewcommand\footrulewidth{0.4pt}

\setlength\parindent{0pt}

%
% Create Problem Sections
%

\newcommand{\enterProblemHeader}[1]{
    \nobreak\extramarks{}{Problem \arabic{#1} continued on next page\ldots}\nobreak{}
    \nobreak\extramarks{Problem \arabic{#1} (continued)}{Problem \arabic{#1}
    continued on next page\ldots}\nobreak{}
}

\newcommand{\exitProblemHeader}[1]{
    \nobreak\extramarks{Problem \arabic{#1} (continued)}{Problem \arabic{#1}
    continued on next page\ldots}\nobreak{}
    \stepcounter{#1}
    \nobreak\extramarks{Problem \arabic{#1}}{}\nobreak{}
}

\setcounter{secnumdepth}{0}
\newcounter{partCounter}
\newcounter{homeworkProblemCounter}
\setcounter{homeworkProblemCounter}{1}
\nobreak\extramarks{Problem \arabic{homeworkProblemCounter}}{}\nobreak{}

%
% Homework Problem Environment
%

\newenvironment{homeworkProblem}[1]{
    \section{Problem \arabic{homeworkProblemCounter}{#1}}
    \setcounter{partCounter}{1}
    \enterProblemHeader{homeworkProblemCounter}
}{
    \exitProblemHeader{homeworkProblemCounter}
}

%
% Homework Details
%   - Title
%   - Due date
%   - Class
%   - Section/Time
%   - Instructor
%   - Author
%

\newcommand{\hmwkTitle}{Homework\ 11}
\newcommand{\hmwkDueDate}{November 1, 2022}
\newcommand{\hmwkClass}{MATH 210}
\newcommand{\hmwkClassTime}{1:00pm}
\newcommand{\hmwkClassInstructor}{Professor Smith}
\newcommand{\hmwkAuthorName}{\textbf{Ryan Magdaleno}}
\newcommand{\hwline}{\begin{center}\line(1,0){358px}\end{center}}

%
% Title Page
%

\title{
    \vspace{2in}
    \textmd{\textbf{\hmwkClass:\ \hmwkTitle}}\\
    \normalsize\vspace{0.1in}\small{Due\ on\ \hmwkDueDate\ at 11:59pm}\\
    \vspace{0.1in}\large{\textit{\hmwkClassInstructor\ \hmwkClassTime}}
    \vspace{3in}
}

\author{\hmwkAuthorName\\\href{mailto:rmagd2@uic.edu}{rmagd2@uic.edu}}
\date{}

\renewcommand{\part}[1]{\textbf{\large Part \Alph{partCounter}}
\stepcounter{partCounter}\\}
%
% Various Helper Commands
%

% Useful for algorithms
\newcommand{\alg}[1]{\textsc{\bfseries \footnotesize #1}}

% For derivatives
\newcommand{\deriv}[1]{\frac{\mathrm{d}}{\mathrm{d}x} (#1)}

% For partial derivatives
\newcommand{\pderiv}[2]{\frac{\partial}{\partial #1} (#2)}

% Integral ds
\newcommand{\dx}{\mathrm{d}x}
\newcommand{\D}[1]{\mathrm{d}#1}

% Image insertion
\newcommand{\img}[2]{\begin{center}\includegraphics[scale=#1]{#2}\end{center}}

% Alias for the Solution section header
\newcommand{\solution}{\textbf{\large Solution}\\}

% Probability commands: Expectation, Variance, Covariance, Bias
\newcommand{\E}{\mathrm{E}}
\newcommand{\Var}{\mathrm{Var}}
\newcommand{\Cov}{\mathrm{Cov}}
\newcommand{\Bias}{\mathrm{Bias}}

\begin{document}

\maketitle

\pagebreak

%%%%%%%%%%%%%%%%%%%%%%%%%%%%%%%%%%%%%%%%%%%%%%%%%%%%%%%%%%%%%%%%%%%%%%%%%%%%%%%%%%%%%%%%%

\begin{homeworkProblem}{}
    \hspace{55pt}\(U = \{0\le z\le36 - x^2-y^2\)\}, \(x = r\cos(\theta)\), 
    \(y = r\sin(\theta)\)
    \[
        \int_{0}^{2\pi}\hspace{-9pt}\int_{0}^{6}\hspace{-5pt}\int_{0}^{36-r^2}
        r\ \D{z}\D{r}\D{\theta}
    \]
    \hwline\solution
    \vspace{-5pt}
    \begin{align*}
        \int_{0}^{36-r^2} r\ \D{z} &= rz\bigg|_0^{36-r^2} = 36r-r^3 \\
        \int_{0}^{6}36r-r^3\ \D{r} &= 18r^2 - \frac{r^4}{4}\bigg|_0^6 = 648 - 324 = 324
        \\ \int_{0}^{2\pi} 324\ \D{\theta} &= 324\theta\bigg|_0^{2\pi} = \boxed{648\pi}
    \end{align*}
\end{homeworkProblem}

\hwline

%%%%%%%%%%%%%%%%%%%%%%%%%%%%%%%%%%%%%%%%%%%%%%%%%%%%%%%%%%%%%%%%%%%%%%%%%%%%%%%%%%%%%%%%%

\begin{homeworkProblem}{}
    \hspace{55pt}\(U = \{0\le x^2 + y^2\le 1, 0\le z\le 5-x-y\}\)
    \[
        \int_{0}^{2\pi}\hspace{-5pt}\int_{0}^{1}
        \hspace{-3pt}\int_{0}^{5-r\cos(\theta)-r\sin(\theta)}
        r\ \D{z}\D{r}\D{\theta}
    \]
    \hwline\solution
    \vspace{-5pt}
    \begin{align*}
        \int_{0}^{5-r\cos(\theta)-r\sin(\theta)}r\ \D{z} &= 
        rz\bigg|_0^{5-r\cos(\theta)-r\sin(\theta)}
        = 5r-r^2\cos(\theta)-
        r^2\sin(\theta) \\
        \int_{0}^{1}5r-r^2\cos(\theta)-r^2\sin(\theta)\ \D{r} &= \frac{5r^2}{2}-
        \frac{r^2}{3}\cos(\theta) - \frac{r^2}{3}\sin(\theta)\bigg|_0^1 \\
        \int_{0}^{2\pi}\frac{5}{2} - \frac{\cos(\theta)}{3} - \frac{\sin(\theta)}{3}
        \ \D{\theta} &= 5\pi - 0 - 0 = \boxed{5\pi}
    \end{align*}
\end{homeworkProblem}

\pagebreak

%%%%%%%%%%%%%%%%%%%%%%%%%%%%%%%%%%%%%%%%%%%%%%%%%%%%%%%%%%%%%%%%%%%%%%%%%%%%%%%%%%%%%%%%%

\begin{homeworkProblem}{}
    \hspace{55pt}\(0\le\rho\le 3\), \(0\le\theta\le\frac{\pi}{2}\),
    \(0\le\phi\le\frac{\pi}{2}\), \(y = \rho\sin(\theta)\sin(\phi)\),
    1st octant.
    \[
        \int\hspace{-5pt}\int\hspace{-5pt}\int_V y\ \D{V}
    \]
    \hwline\solution
    \vspace{-5pt}
    \begin{align*}
        \int_{0}^{\frac{\pi}{2}}\hspace{-5pt}\int_{0}^{\frac{\pi}{2}}
        \hspace{-5pt}\int_{0}^{3} \rho\sin(\theta)\sin(\phi)\rho^2\sin(\phi)
        \ \D{\rho}\D{\phi}\D{\theta} &=
        \int_{0}^{\frac{\pi}{2}}\hspace{-5pt}\int_{0}^{\frac{\pi}{2}}
        \hspace{-5pt}\int_{0}^{3} \sin^2 (\phi)\sin(\theta)\rho^3
        \ \D{\rho}\D{\phi}\D{\theta} \\
        \int\rho^3\ \D{\rho} &= \frac{\rho^4}{4}\bigg|^3_0 = \frac{81}{4} \\
        \frac{81}{4}\cdot \int_{0}^{\frac{\pi}{2}}
        \hspace{-5pt}\int_{0}^{\frac{\pi}{2}} \frac{\sin^2(\phi)\sin(\theta)}
        {4}\ \D{\phi}\D{\theta} &= \int\sin^2 (\phi)\ \D{\phi} \\
        \frac{81\sin(\theta)\phi}{8} - \frac{81\sin(\theta)\sin(2\phi)}{16} &=
        \frac{81\sin(\theta)\left(\phi -\frac{\sin(2\phi)}{2}\right)}{8}\Bigg|^0_{\frac
        {\pi}{2}} \\
        &= \int_{0}^{\frac{\pi}{2}}\frac{81\pi\sin(\theta)}{16}\ \D{\theta} = -\cos
        (\theta) \\
        -\frac{81\pi\cos(\theta)}{16}\bigg|^{\frac{\pi}{2}}_0 &= 0 -
        \left(-\frac{81\pi}{16}\right) = \boxed{\frac{81\pi}{16}}
    \end{align*}
\end{homeworkProblem}

%%%%%%%%%%%%%%%%%%%%%%%%%%%%%%%%%%%%%%%%%%%%%%%%%%%%%%%%%%%%%%%%%%%%%%%%%%%%%%%%%%%%%%%%%

\end{document}