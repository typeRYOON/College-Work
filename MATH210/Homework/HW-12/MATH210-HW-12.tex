\documentclass{article}

\usepackage[bookmarksnumbered, colorlinks, plainpages]{hyperref}
\usepackage{fancyhdr}
\usepackage{extramarks}
\usepackage{amsmath}
\usepackage{amsthm}
\usepackage{amsfonts}
\usepackage{tikz}
\usepackage[plain]{algorithm}
\usepackage{algpseudocode}
\usepackage{graphicx}
\usepackage{nicefrac}

\graphicspath{ {./img/} }
\usetikzlibrary{automata, positioning}

%
% Basic Document Settings
%

\topmargin=-0.45in
\evensidemargin=0in
\oddsidemargin=0in
\textwidth=6.5in
\textheight=9.0in
\headsep=0.25in

\linespread{1.1}

\pagestyle{fancy}
\lhead{\hmwkAuthorName}
\chead{\hmwkClass\ (\hmwkClassInstructor\ \hmwkClassTime): \hmwkTitle}
\rhead{\firstxmark}
\lfoot{\lastxmark}
\cfoot{\thepage}

\renewcommand\headrulewidth{0.4pt}
\renewcommand\footrulewidth{0.4pt}

\setlength\parindent{0pt}

%
% Create Problem Sections
%

\newcommand{\enterProblemHeader}[1]{
    \nobreak\extramarks{}{Problem \arabic{#1} continued on next page\ldots}\nobreak{}
    \nobreak\extramarks{Problem \arabic{#1} (continued)}{Problem \arabic{#1}
    continued on next page\ldots}\nobreak{}
}

\newcommand{\exitProblemHeader}[1]{
    \nobreak\extramarks{Problem \arabic{#1} (continued)}{Problem \arabic{#1}
    continued on next page\ldots}\nobreak{}
    \stepcounter{#1}
    \nobreak\extramarks{Problem \arabic{#1}}{}\nobreak{}
}

\setcounter{secnumdepth}{0}
\newcounter{partCounter}
\newcounter{homeworkProblemCounter}
\setcounter{homeworkProblemCounter}{1}
\nobreak\extramarks{Problem \arabic{homeworkProblemCounter}}{}\nobreak{}

%
% Homework Problem Environment
%

\newenvironment{homeworkProblem}[1]{
    \section{Problem \arabic{homeworkProblemCounter}{#1}}
    \setcounter{partCounter}{1}
    \enterProblemHeader{homeworkProblemCounter}
}{
    \exitProblemHeader{homeworkProblemCounter}
}

%
% Homework Details
%   - Title
%   - Due date
%   - Class
%   - Section/Time
%   - Instructor
%   - Author
%

\newcommand{\hmwkTitle}{Homework\ 12}
\newcommand{\hmwkDueDate}{November 2, 2024}
\newcommand{\hmwkClass}{MATH 210}
\newcommand{\hmwkClassTime}{1:00pm}
\newcommand{\hmwkClassInstructor}{Professor Smith}
\newcommand{\hmwkAuthorName}{\textbf{Ryan Magdaleno}}
\newcommand{\hwline}{\begin{center}\line(1,0){358px}\end{center}}

%
% Title Page
%

\title{
    \vspace{2in}
    \textmd{\textbf{\hmwkClass:\ \hmwkTitle}}\\
    \normalsize\vspace{0.1in}\small{Due\ on\ \hmwkDueDate\ at 8:00am}\\
    \vspace{0.1in}\large{\textit{\hmwkClassInstructor\ \hmwkClassTime}}
    \vspace{3in}
}

\author{\hmwkAuthorName\\\href{mailto:rmagd2@uic.edu}{rmagd2@uic.edu}}
\date{}

\renewcommand{\part}[1]{\textbf{\large Part \Alph{partCounter}}
\stepcounter{partCounter}\\}
%
% Various Helper Commands
%

% Useful for algorithms
\newcommand{\alg}[1]{\textsc{\bfseries \footnotesize #1}}

% For derivatives
\newcommand{\deriv}[1]{\frac{\mathrm{d}}{\mathrm{d}x} (#1)}

% For partial derivatives
\newcommand{\pderiv}[2]{\frac{\partial}{\partial #1} (#2)}

% Integral ds
\newcommand{\dx}{\mathrm{d}x}
\newcommand{\D}[1]{\mathrm{d}#1}

% Image insertion
\newcommand{\img}[2]{\begin{center}\includegraphics[scale=#1]{#2}\end{center}}

% Alias for the Solution section header
\newcommand{\solution}{\textbf{\large Solution}\\}

% Probability commands: Expectation, Variance, Covariance, Bias
\newcommand{\E}{\mathrm{E}}
\newcommand{\Var}{\mathrm{Var}}
\newcommand{\Cov}{\mathrm{Cov}}
\newcommand{\Bias}{\mathrm{Bias}}

\begin{document}

\maketitle

\pagebreak

%%%%%%%%%%%%%%%%%%%%%%%%%%%%%%%%%%%%%%%%%%%%%%%%%%%%%%%%%%%%%%%%%%%%%%%%%%%%%%%%%%%%%%%%%

\begin{homeworkProblem}{}
    \hspace{55pt}\(0\le \rho\le R\), \(0\le\theta\le 2\pi\), \(0\le\phi\le\pi\)
    \[
        \int_{0}^{\pi}\hspace{-3pt}\int_{0}^{2\pi}\hspace{-7pt}\int_{0}^{R}
        \rho^2\sin(\phi)\ \D{\rho}\D{\theta}\D{\phi}
    \]
    \hwline\solution
    \begin{align*}
        \int_{0}^{R}\rho^2\ \D{\rho} &= \frac{\rho^3}{3}\bigg|^R_0 
        = \int_{0}^{2\pi}\frac{R^3}{3}\sin(\phi)\ \D{\theta}
        = \frac{2\pi R^3\sin(\phi)}{3} \\
        \frac{2\pi R^3}{3}\cdot\int_{0}^{\pi}\sin(\phi)\ \D{\phi} &=
        -\frac{2\pi R^2\cos(\phi)}{3}\bigg|^{\phi=\pi}_0 \\
        &= -\frac{2\pi R^3(-1)}{3} + \frac{2\pi R^3(1)}{3} \\
        &= \boxed{\frac{4\pi R^3}{3}}
    \end{align*}
\end{homeworkProblem}

\hwline

%%%%%%%%%%%%%%%%%%%%%%%%%%%%%%%%%%%%%%%%%%%%%%%%%%%%%%%%%%%%%%%%%%%%%%%%%%%%%%%%%%%%%%%%%

\begin{homeworkProblem}{}
    \hspace{55pt}\(0\le \rho\le 2\), \(0\le\theta\le 2\pi\), 
    \(0\le\phi\le \nicefrac{\pi}{6}\)
    \[
        \int_{0}^{\frac{\pi}{6}}\hspace{-5pt}\int_{0}^{2\pi}\hspace{-7pt}\int_{0}^{2}
        \rho^2\sin(\phi)\,\,\D{\rho}\D{\theta}\D{\phi}
    \]
    \hwline\solution
    \begin{align*}
        \int_{0}^{2}\rho^2\ \D{\rho} &= \frac{\rho^3}{3}\bigg|^2_0 = \frac{8}{3} \\
        \int_{0}^{2\pi} 1\ \D{\theta} &= 2\pi \\
        \int_{0}^{\frac{\pi}{6}} \frac{16\pi\sin(\phi)}{3}\ \D{\phi} &=
        \frac{16\pi}{3}\cdot\int\sin(\phi)\ \D{\phi} = \cos(\nicefrac{\pi}{6}) =
        \frac{\sqrt{3}}{2} \\
        \frac{16\pi\cos(\phi)}{3}\bigg|_0^{\phi=\nicefrac{\pi}{6}} &= 
        -\frac{16\pi\frac{\sqrt{3}}{2} + 16\pi}{3} \\
        &= \boxed{-\frac{8\sqrt{3}\pi + 16\pi}{3}}
    \end{align*}
\end{homeworkProblem}

\pagebreak

%%%%%%%%%%%%%%%%%%%%%%%%%%%%%%%%%%%%%%%%%%%%%%%%%%%%%%%%%%%%%%%%%%%%%%%%%%%%%%%%%%%%%%%%%

\begin{homeworkProblem}{}
    \hspace{55pt}\(x = \frac{u}{v + 5}\), \(y = \frac{uv}{v+5}\)
    \[
        \int\hspace{-5pt}\int_D 5x+y\ \D{A}
    \]
    \hwline\solution
    \(5x+y=3\), \(5x + y = 6\), \(y = x\), \(y=2x\)
    \begin{align*}
        (5x + y = 3) &= \left(5\left(\frac{u}{v+5}\right) + \frac{uv}{v+5} = 3\right) \\
        \frac{u(5+v)}{v+5} = 3 &\text{ or } u=3 \\
        5x + y &= 6,\,u=6 \\
        y=x: \frac{uv}{v+5} &= \frac{u}{v+5},\ uv = u,\ v = 1 \\
        y=2x: \left(\frac{uv}{v+5} = 2\left(\frac{u}{v+5}\right)\right) &= 
        \frac{uv}{u} = \frac{2u}{u},\ v = 2 \\
        u = 3,\ u = 6,\ v = 1,\ v &= 2,\ [3,6]\times[1,2] \\
        \frac{\partial x}{\partial u}\cdot\frac{\partial y}{\partial v}
        &- \frac{\partial x}{\partial v}\cdot \frac{\partial y}{\partial u}
    \end{align*}
    \[
        \begin{bmatrix}
            x_u & x_v
            \\[1ex]
            y_u & y_v
        \end{bmatrix} =
        \begin{bmatrix}
            \frac{1}{v+5} 
            & \frac{-u}{(v+5)^2}
            \\[1ex]
            \frac{v}{v+5}
            & \frac{5u}{(v+5)^2}
        \end{bmatrix}
    \]
    \begin{align*}
        &= \left(\frac{1}{v+5}\cdot\frac{5u}{(v+5)^2}\right) -
        \left(\frac{-u}{(v+5)^2}\cdot\frac{v}{v+5}\right) \\
        &= \frac{5u}{(v+5)^3} - \frac{-uv}{(v+5)^3} \\
        &= \frac{5u+uv}{(v+5)^3} \\
        \text{Jacobian} &= \frac{u}{(v+5)^2} \\
        5x + y &= 3 \text{ or } u \\
        \int_{1}^{2}\hspace{-5pt}\int_{3}^{6}\frac{u^2}{(v+5)^2}\ \D{u}\D{v} &=
        \int_{3}^{6}u^2\ \D{u}\cdot \int_{1}^{2}\frac{1}{(v+5)^2}\ \D{v} \\
        \int_{3}^{6}u^2\ \D{u} &= \frac{u^3}{3}\bigg|_3^6 = 72-9 = 63 \\
        \int_{1}^{2}\frac{1}{(v+5)^2}\ \D{v} &= \frac{-1}{v+5}\bigg|^2_1 = \frac{1}{42} \\
        63\cdot \frac{1}{42} &= \frac{63}{42} = \boxed{\frac{3}{2}}
    \end{align*}
\end{homeworkProblem}

%%%%%%%%%%%%%%%%%%%%%%%%%%%%%%%%%%%%%%%%%%%%%%%%%%%%%%%%%%%%%%%%%%%%%%%%%%%%%%%%%%%%%%%%%

\end{document}
